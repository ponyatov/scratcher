\documentclass{magazine}

\begin{document}


\begin{titlepage}

\begin{multicols}{2}

\noindent\includegraphics[width=0.9\columnwidth]{logo/chbz.png}

\columnbreak

{\Huge \Scr \#00}\bigskip

\noindent\textsf{Online журнал для \scr ов\ --- людей, чье хобби создавать вещи
и технологии по следам уже существующих, в сотый раз изобретать велосипед, чтобы
разобраться как оно работает, научиться делать самому, а возможно найти новый
или забытый способ что-то сделать, и конечно получить удовольствие от процесса
поиска.}

\end{multicols}

\begin{center}
{\Huge\textbf{Кустарь-одиночка с мотором}}
{\large\bigskip}
\end{center}

\noindent\includegraphics[width=\textwidth]{logo/trolley.jpg}

\bigskip
Редакция: $<$\href{mailto:dponyatov@gmail.com}{dponyatov@gmail.com}$>$

\bigskip
\href{https://github.com/ponyatov/scratcher}{\includegraphics[height=5em]{logo/GitHub.png}}
\ \includegraphics[height=3em]{logo/linuxpowered.png}
\ \includegraphics[height=5em]{logo/OpenHardware.png}
{\Large Powered by \LaTeX}

\end{titlepage}

\tableofcontents\clearpage



\input{00/intro}

\article{КопиПаста}{Как сказать "Начать с нуля"\,?}{

\begin{itemize}
  \item
На английском — to start from scratch; to start over.
  \item
На испанском — empezar de cero; empezar de nuevo.
  \item
На итальянском — partire dal niente.
\end{itemize}

В общем-то во всех языках мы видим <<кальку>>, выделяется только одно,
содержащее слово <<scratch>> (царапина, черта).

С момента его возникновения, это выражение немного поменяло свое значение.
Сейчас оно используется, когда мы хотим сказать <<начать снова, начать с
начала>> в том смысле, что мы потерпели поражение при первой попытке.

Фраза родилась в конце 19-го века и тогда просто значила <<начинать без
преимуществ>>. Слово <<scratch>> использовалось с 18-го века как спортивный
термин, обозначающий линию старта, прочерченную на земле. Впервые такая линия
упоминалась в описании игры в крикет\ --- на ней стоял игрок, отбивающий мяч.

<<Start from scratch>> в качестве понятия <<начинать с нуля>> пришло к нам из
бокса. Прочерченная линия определяла позиции боксеров, когда они стояли друг
напротив друга в начале поединка. Отсюда также произошло выражение <<up to
scratch>>, (быть на должной высоте, в прекрасной форме), т.е. соответствовать
стандартам, предъявляемым боксерам, делающим заявку на матч.

Позднее <<scratch>> стали называть любую стартовую точку в бегах. Термин стали
использовать в <<гандикап>>-соревнованиях (handicap), в которых более слабый
участник получает фору. Например, в велоспорте те, у кого нет преимуществ, стоят
на линии, в то время как остальные стоят впереди. Другие виды спорта, особенно
гольф, заимствовали переносное значение <<scratch>> как термин для обозначения
<<без преимуществ\ --- начинать с нуля>>.

В The Fort Wayne Gazette (апрель 1887) содержится самое раннее упоминание
<<start from scratch>> — в репортаже о <<‘no-handicap>> велосипедной гонке:

<<It was no handicap. Every man was qualified to and did start from scratch.>>

По моим наблюдениям, <<start from scratch>> употребляется чаще в письменной речи
(например, уже несколько раз видела его в статьях в интернете), а <<to start
over>>\ --- в разговорной (слышала в американском сериале).

}{\copyright\ \href{http://www.lingvaroom.ru/kak-skazat-nachat-s-nulya/}{Юлия
Горбунова}}



\article{КопиПаста}{Персональное производство\\еще один шаг к реконизму}{

Один из важных моментов в построении реконистической экономики — это
трансформация традиционного, корпоративного производства, основанного на
обязательной организации, как в смысле объединения людей, средств производства,
финансовых и материальных ресурсов, так и в смысле появления так называемых
юридических лиц как практически единственных субъектов производства. Такое
производство в значительной части сфер деятельности будет вытесняться
индивидуальным производством, когда  любой желающий, используя так называемые
микрофабрики — миниатюрный комплект универсального оборудования, сможет
производить достаточно широкую линейку продукции, как для личного пользования,
так и для продажи. Произойдет нечто вроде возврата к ремесленному производству
средневековья и даже к натуральному хозяйству, но на неизмеримо более высоком
технологическом уровне. Особую ценность в таких условиях обретет информация —
продаваться будет не товар, а инструкция для микрофабрики, как данный товар
изготовить. Конечно, такие инструкции будет не только продаваться, но и
распространяться бесплатно, а также вороваться. Разумеется, это серьезно
поменяет привычную нам социально-экономическую систему.

В последнем номере журнала «Наука и жизнь» (№8 за 2012 год), появилась небольшая
заметка, в которой рассказывается о разработке профессора Массачусетского
технологического института Нила Гершенфельда, который предложил концепцию
миниатюрной фабрики-лаборатории (Fab Lab). Фабрика-ла\-бо\-ра\-то\-рия
представляет собой комплекс станков, совместно работающих под управлением
персонального компьютера. Идея Гершенфельда получила широкое распространение и
десятки университетов и исследовательских центров экспериментируют с такими
мини-фабриками. В России первая такая фабрика создана в Московском институте
стали сплавов, в ее составе фрезерный станок для обработки древесины, пластиков
и мягких металлов, гравировальный прецизионный станок для производства печатных
плат, установка лазерной резки, плоттер для раскроя гибких материалов и
производства гибких микросхем, и 3D-принтер, предназначенный для изготовления
любых изделий из ABS-пластика.

Так что, возможно, что лет через десять, для того чтобы поменять надоевший
мобильный телефон, мы будем заходить на сайт какой-нибудь Нокии, скачивать файл
с данными, запускать его в программе на домашнем компьютере, а стоящий на
тумбочке агрегат, очертаниями смахивающий на современное МФУ, погудев пару
минут, выбросит в приемный лоток еще горячую, пахнущую свежим пластиком
мобилку\ldots
\smiley

}{\copyright\
\href{http://blog42.ws/personalnoe-proizvodstvo-eshhe-odin-shag-k-rekonizmu/}{AG}}



\article{Новости технологий}{Злобный янки в
3D-танке\\Полевые 3D-принтеры на службе американской армии}{

Пока специалисты в области 3D-печати рассуждают о перспективах приенения
технологии, а энтузиасты осторожно говорят о потенциальной возможности печати
необходимого скарба сразу на лунной базе (чтобы не тащить лишнее с Земли),
американская армия без всяких промедлений нашла применение 3D-печати уже сейчас.
Военные США стали использовать мобильные лаборатории Expeditionary Lab Mobile с
3D-принтерами в комплекте.

Основными задачами лабораторий Expeditionary Lab Mobile (сокращённо — ELM) будет
изготовление одноразовых инструментов для нужд армии, а также внесение
корректирующих дополнений в уже существующее оборудование — «полевое»
использование часто требует определённой доводки. В качестве примера приводится
случай, когда войска получают партию карманных фонарей с дефектом – быстро
выходящим из строя предохранителем выключателя. Находясь в кармане у военного,
такой фонарь может самопроизвольно включиться и либо выдать местонахождение
бойца, либо впустую разрядить батарейки. Однако, имея под рукой ELM, можно
быстро допечатать предохранители, без необходимости отсылки всей партии обратно
в США для замены.

\noindent\includegraphics[width=\textwidth]{00/fig/yanky01.jpg}
\noindent\textbf{Чебураторы на тропе войны}

\begin{multicols}{2}
\noindent\includegraphics[width=0.8\columnwidth]{00/fig/yanky03.jpg}
\columnbreak
\noindent\includegraphics[width=0.8\columnwidth]{00/fig/yanky02.jpg}
\end{multicols}

Ещё одним примером можно назвать реальный случай недоработки в конструкции
миноискателя, приведший к тому, что время работы прибора из-за иракской жары
сократилось с восьми часов до 45 минут. В результате во время многодневных
миссий солдаты были вынуждены носить большое количество дополнительных батарей.
Использование ELM позволило сконструировать адаптер для использования батарей
другого типа и увеличить время работы миноискателя до девяти часов.

Expeditionary Lab Mobile представляет собой стандартный грузовой контейнер
(6,1$\times$2,4 м), внутри которого находятся 3D-принтер, специальные станки с
ЧПУ (для изготовления более сложных деталей из стали и алюминия) и набор
традиционных инструментов: резак, сварочный аппарат, циркулярная пила,
маршрутизатор, лобзик и сабельная пила. Кроме того, в комплекте ELM имеется
спутниковое оборудование связи для проведения телеконференций с чиновниками и
инженерами в США – для оперативных корректировок работы. При каждой лаборатории
будут находиться два инженера. Все лаборатории будут связаны между собой единой
компьютерной сетью.

Стоит отметить, что подобный способ изготовления износившихся или недостающих
деталей довольно дорог: стоимость каждой лаборатории составляет около 2,8
миллиона долларов. Планируется, что первые ELM будут испытаны в Афганистане.
Кроме того, можно надеяться, что успешное применение новых технологий на <<поле
боя>> будет способствовать их внедрению для мирных операций. Например, во время
стихийных бедствий.

}{\copyright\
\href{http://www.computerra.ru/50860/polevyie-3d-printeryi-na-sluzhbe-amerikans/}{Компьютерра, Николай Маслухин}
}




\article{Принципы скрэтчера}{Чем оно отличается от прочего DIY}{

\begin{itemize}

\item Из говна и палок

Чем больше г и кривее палки, тем круче \scr

\item Сделай сам, расскажи другим

Необходим активный обмен информацией для мимимизации и так больших расходов на
избретение колес

\item Минимум покупных изделий

В идеале изготовление всего из чисто природных материлов и без стартового
инструмента

\item Все покупные ништяки должны быть всегда доступны в любом ближайшем
магазине

Чтобы каждый мог легко и быстро повторить понравившийся хак.

\emph{Следует обратить внимание, что этому принципу противоречит использование
техно-мусора, различных деталей от старой техники и т.п.\ --- вот сколько
сейчас у вас например сломанных стиралок, или дохлых телевизоров в доме ?}

\emph{Еще одно противоречие\ --- покупка комплектухи по почте в Китае, и заказ
редких компонентов в магазинах}

\item Покупаться должны \textbf{самые дешевые}\ и самые кривые
комплектующие

Но при этом не нужно скатываться на использование раритета\ --- см. доступность.

\item Приоритетно использование более ранней ступени
\term{технологического передела}

Например вместо использование готового заводского сверла взять хвостовик от
сломанного, и выпилить сверло самому. Правильнее было бы взять твердосплавную
заготовку, но это противоречит принципу доступности, т.к. их нет в
доступных магазинах. Вариант использование куска проката из инструментального
сплава лучше, потому что можно еще повыделываться с термичкой \smiley

\item Должно использоваться \term{открытое программное обеспечение}

Причем написанное целиком самостоятельно на ассемблере, ну или хотя бы собрать
\href{http://cross-lfs.org/}{Cross Linux From Scratch}\ для DIY компьютера,
спаянного из отдельных деталей с помощью самодельного паяльника.

В процессе неплохо попутно изобрести пару уникальных языков программирования,
написать на них операционную систему и комплект программного обеспечения.

\item Желательно использовать нетиповые приемы работы и технологии

\item При разработке конструкций нужно стремиться использовать малоизвестные и
уникальные конструктивные решения

\item Максимум самодельного инструмента

\item Идеал \scr а\ --- пройти всю технологическую цепочку от каменного
рубила до обрабатывающего центра с ЧПУ

И с разгона заскочить еще дальше, обогнав текущие лабораторные разработки по
3D-печати, зональной плавке и прочим свежакам технологии

\item Минимум повторов готовых изделий и унификации

Каждая поделка должна быть прекрасна в своей уникальности, и ее область
применения должна быть максимально узкозаточенной под ваши задачи. Применение
унификации, общеизвестных конструктивных решений и принципов работы неприемлемо,
т.к. какой смысл повторять уже готовое изделие, которое можно купить ?!

\item Больше науки

Копайте книги по математике, физике и химии, больше статей и техрасчетов.
Чем больше матана и самопала, тем выше левел. Не забывайте про пропагандизм
достижений на форумах (особенно нетематических) и в оффлайне.

\item Больше синей изоленты

\item Обязательно используйте ардуину

Даже если устройство вообще не предполагает использование электричества\ ---
прикрутите микроконтроллер изолентой, и подключите к нему компьютер.

\item Для успеха проекта обязательно нужен ковер

\item На демонстрационном видео должно что-нибудь отвалиться или чпохнуть
волшебным синим дымом

\end{itemize}

Набор принципов \scr а выглядит похоже на инструкцию <<Как просрать полимеры>>,
поэтому как и в любом другом деле, не нужно доводить их исполнение до фанатизма.
\emph{Новизна и уникальность} на первом месте, и не надо забывать что это все же
хобби, а не жизненная миссия. Нужно всего лишь следить за соблюдением баланса
между потраченными средствами, временем, и полученным от процесса удовольствием.

Главное достоинство отработанных вещей и технологий\ --- на их доводку и
проверку уже было потрачено гигантское количество ресурсов. Самодельные аналоги
в любом случае будут хуже и на порядок дороже, чем серийное изделие, за редким
исключением узконишевого использования, для которого готовое решение почему-то
не подходит.

Из положительных эффектов \scr ства можно отметить хорошие общетехнические
знания, и умение при необходимости быстро слепить <<костыль>> (временное решение
проблемы) из подручных ресурсов.

}{}




\article{Инструмент}{Кустарь-одиночка с мотором}{

\href{https://www.youtube.com/playlist?list=PL6mXlWgvuzAu4jMVzOOKXXDVF3r1yAaz4}{Отличная
серия видео по изготовлению токарника}\bigskip

Возникает вопрос, где взять самый дешевый, легко доставаемый и универсальный
электропривод, и нужно ли вообще пользоватся электричеством.

Вопрос по использованию электричества оставляем самым упоротым \scr ам 80-го
левела. Будем исходить из того, что хоть какое-то (под нагрузку мощностью от
100$\div$200\,Вт) сетевое электричество доступно сейчас всем, кроме туристов,
огородников и прочих полевиков, не укомплектованных бензогенератором.

\emph{Соответственно в комплекте базового инструмента предполагаем наличие
минимум электродрели и паяльника.}

\subarticle{Электроинструмент}

\subsubarticle{Дрелъ}

\begin{multicols}{2}
\noindent\href{http://leroymerlin.ru/catalogue/instrumenty/elektroinstrument/dreli\_udarnye/13805983/}{
\includegraphics[width=\columnwidth]{00/fig/PraktylR.jpg}}
\textbf{Дрель ударная сетевая Praktyl-R PID13D01 400\,Вт (!)395\,р.}

\columnbreak

\noindent\href{http://leroymerlin.ru/catalogue/instrumenty/elektroinstrument/dreli\_bezudarnye/11857763/}{
\includegraphics[width=\columnwidth]{00/fig/D_11_530ER.jpg}}
\textbf{Дрель безударная сетевая Интерскол Д-11/530ЭР (с БЗП) 1120\,р.}
\end{multicols}

Дрель\ --- одноразовая китайчатина от 400\,р. Цена крайне низкая, поэтому в
целях тестирования взял один экземпляр на натурные испытания, результаты по
живучести будут в следующих номерах. Подаются уже брендированные на Леруа
Мерлен, наклейка <<PID13D01 Ударная дрель 400\,Вт, 13\,мм>>. Скорость
регулируется глубиной нажатия курка, крутилка на курке ограничивает глубину
механически, фиксатор держит скорость близко к минимальной, запаха горелой
пластмассы через несколько минут работы на холостом ходу нет.

По надежности рекомедуется Интерскол
1100+\,р. Надежность Интерскола\ --- не <<китай>>, классика ДУ-580ЭР
работает в хвост и гриву в университете ежедневно с $\sim$2005 г., используется
криворукими студентами, лежит в подвале в пыли от точила, и никаких вопросов
даже со щетками.

Если не планируете много сверлить бетон, \textbf{берите дрель без ударного
механизма}: отсутствуют лишние продольные перемещения, что может быть важно при
использовании в качестве шпинделя сверлильного станка, и механизации других
технологических поделок.

\emph{Шуруповерт\ --- буржуйство, у него нет 43\,мм шейки для фиксации, поэтому
как средство электропривода он практически бесполезен, и нужен собственно для
заворачивания большого количества саморезов. Хотя наличие ограничителя
крутящего момента и малые габариты удобны при сверлении и сборке поделок.}

\subsubarticle{Лобзик}

\begin{multicols}{2}
\noindent\href{http://leroymerlin.ru/catalogue/instrumenty/elektroinstrument/lobziki/13805991/}{
\includegraphics[width=\columnwidth]{00/fig/LobzPraktyl.jpg}}
\textbf{Лобзик Praktyl 350 Вт 356\,р.}

\columnbreak

\noindent\href{http://leroymerlin.ru/catalogue/instrumenty/elektroinstrument/lobziki/12114283/}{
\includegraphics[width=\columnwidth]{00/fig/LobzMakita4329.jpg}}
\textbf{Лобзик Makite 4329 2260\,р.}
\end{multicols}

Лобзик опционален, и куда полезнее шуруповерта, китай-хлам 350+\,р, чуть
поприличнее 2000+\,р. \textbf{Не берите с маятником дешевле 5--7\,тыс.р}.

\subarticle{Паяльник}

\begin{multicols}{2}
\noindent\includegraphics[width=\columnwidth]{00/fig/EPSN25.jpg}
\textbf{Паяльник ЭПСН-25/220}

\columnbreak

\noindent\href{http://voltmaster-samara.ru/catalog/product/00067650/}{
\includegraphics[width=\columnwidth]{00/fig/SV-55310-25.jpg}}
\textbf{Паяльник 220В 25Вт, СВЕТОЗАР, SV-55310-25 230\,р.}
\end{multicols}

Паяльник\ --- обязателен дешевый сетевой мощностью не менее 20\,Вт, типа
ЭПСН-25/220. \emph{Ограничитель мощности или регулятор температуры труъ-\scr\
должен собрать самостоятельно.}

Для сборки электроники хорошо также иметь маленький монтажный 12\,В 8\,Вт от
паяльной станции ZD-927 ($\sim$100\,р), без самой станции.

\begin{multicols}{2}
\noindent\href{http://voltmaster-samara.ru/catalog/product/00091478/}{
\includegraphics[width=\columnwidth]{00/fig/ZD-721N.jpg}}
\textbf{Паяльник 220В 25Вт ZD-721N 175\,р.}

\columnbreak

\noindent\href{http://voltmaster-samara.ru/catalog/product/00047380/}{
\includegraphics[width=\columnwidth]{00/fig/Iron8W.jpg}}
\textbf{Паяльник для станции ZD-927 12\,В 8\,Вт 85\,р.}
\end{multicols}

\subsubarticle{Паяльная станция}

\begin{multicols}{2}
Если не жалко 500\,р, берите ZD-927 целиком, внутри простейший регулятор
мощности, и вам не понадобится источник питания на 12\,В, который вы еще не
сделали. \emph{Но труъ путь\ --- конечно собрать свой паяльник целиком, из
нихрома, жала из толстой проволоки или медной шины, и самостоятельно выточенной
ручки.}

\columnbreak

\noindent\href{http://voltmaster-samara.ru/catalog/product/00073790/}{
\includegraphics[width=\columnwidth]{00/fig/ZD927.jpg}}
\textbf{Паяльная станция ZD-927 520\,р.}
\end{multicols}

Паяльные станции типа Lukey 702/853D (3000+\,р) естественно не рассматриваем
\smiley. Для работы или регулярного хобби паяльная станция с феном, а может даже
и встроенным источником питания, вещь незаменимая, и не такая уж дорогая, но для
\scr а слишком технологичная.

\begin{multicols}{2}
\noindent\href{http://voltmaster-samara.ru/catalog/product/00073444/}{
\includegraphics[width=\columnwidth]{00/fig/Lukey702.jpg}}
\textbf{Паяльная станция LUKEY 702 3100\,р.}

\columnbreak

\noindent\href{http://shop.siriust.ru/product\_info.php/cPath/23\_28\_269/products\_id/15290}{
\includegraphics[width=\columnwidth]{00/fig/Lukey853D.jpg}}
\textbf{Паяльная станция LUKEY 853D с источником питания 5200\,р.}

\end{multicols}

\subarticle{Жвигатель}

\begin{multicols}{2}
\noindent\includegraphics[width=\columnwidth]{00/fig/DrelBoren.jpg}

\noindent\includegraphics[width=\columnwidth]{00/fig/DrelShliph.jpg}

\noindent\includegraphics[width=\columnwidth]{00/fig/DrelLathe.jpg}

\noindent\includegraphics[width=\columnwidth]{00/fig/DrelLathe2.jpg}
\end{multicols}

Первый кандидат на место универсального электропривода достается той
самой дрели, не забываем об обязательном наличии 43\,мм монтажной шейки.
Достоинство дрели как привода\ --- прямое подключение к сети, встроенный
редуктор, есть модели с простой регулировкой оборотов, резьба и отверстие под
винт на валу, в комплекте есть патрон для зажима мелких деталей в
точилке\footnote{\ БЗП удобен, патрон с ключем дает лучший зажим и возможно
точнее}

Ограниченно доставаемые двигатели от стиральных машин, отличаются мощностю и
оборотистостью, особенно от старых моделей. Часто доступны сразу с готовым
шкивом на валу, который иногда проще использовать, чем снять.

\bigskip
Автозапчасти: привод печки Камаза, двигатель постоянного тока 24\,В
  50\,Вт

\begin{multicols}{2}
\noindent\includegraphics[width=\columnwidth]{00/fig/VyatkaDvig.jpg}
\textbf{Жвигатель Вятка-Автомат 19??\,г.}

\noindent\includegraphics[width=\columnwidth]{00/fig/KamazDvig.jpg}
\textbf{Двигатель печки Камаза}

\noindent\includegraphics[width=\columnwidth]{00/fig/AIRE.jpg}
\textbf{АИРЕ 56 B2, 0.2\,КВт}

\end{multicols}

\bigskip
Новые асинхронные двигатели АИРЕ 56 B2/B4 (3000/1500 об.)
с заводским конденсатором, подключается к сети $\sim$220\,В, цена от 2500\,р.
С ростом размеров и мощности цена резко повышается.
Следует обратить внимание на возможность монтажа на дополнительный фланцевый
подшипниковый щит, (?) с моделями АИРЕ 80.

\subsubarticle{Фрезерный шпиндель}

Съемные фрезерные шпиндели, поставляются отдельно или в комплекте с
насадкой ручного фрезера по дереву. Лучшие, со стальной шейкой\ --- Kress,
активно применяются хобби-ЧПУшниками.

\begin{multicols}{2}

\noindent\href{http://kress-shop.ru/product/frezernyj-dvigatel-530-fm-kress-06082302/}{
\includegraphics[width=\columnwidth]{00/fig/Kress530.jpg}}
\textbf{Фрезерный двигатель KRESS 530/800/1050 FM(E) 5600+\,р.}

\noindent\includegraphics[width=\columnwidth]{00/fig/Interskol30.jpg}
\textbf{Шпиндель Интерскол ФМ-30/750 /снят с производства/}

\end{multicols}

Попроще и сильно дешевле делал Интерскол, иногда попадается noname.

Недостаток как универсального привода\ --- они высокоскоростные,
возникают проблемы с понижающими передачами. Применение\ --- приводной
высокоскоростной инструмент: боры, фрезы по дереву, микроинструмент для
граверов (микродиски, шарошки).

\begin{multicols}{2}
\noindent\href{http://www.kuvalda.ru/catalog/1867/27920/}{
\includegraphics[width=0.5\columnwidth]{00/fig/InterskolFM55.jpg}}
\textbf{Фрезер сетевой Интерскол ФМ-55/1000 Э 5050\,р.}

\columnclear

\noindent\includegraphics[width=0.5\columnwidth]{00/fig/ER11.jpg}
\end{multicols}

Китайские воздушные шпиндели постоянного тока с цанговыми патронами ER11:
привод для сверлилки и микроинструмента. Требуют источник питания
постоянного тока 9$\div$48\,В.

\subarticle{Ручной инструмент}

По мелкому ручному инструменту вопрос открыт.

Учитывая доступность и наличие дешевых вариантов, стоит ли использовать старый
опыт мастеров-ремесленников, когда ученику давали только напильник, и он сам
должен был изготовить себе весь инструмент\,?

По крайней мере, вот этот вариант точно не подходит \smiley:

\noindent\includegraphics[width=\columnwidth]{00/fig/PK5308BM.jpg}
\textbf{PK-5308BM универсальный набор инструментов Pro'sKit}

Но пара надфилей, заточной камень на дрель, комплект сверел и несколько листов
наждачки вполне допускаются \smiley.

Если хочется посложнее, можно ограничиться только парой электродвигателей:
\begin{enumerate}
  \item
относительно медленный высокомоментный АИРЕ 56 B2/4 на силовой привод и
\item
высокоскоростной 10+ тыс.об$^{-1}$ для сверления, шлифования насадками и т.п.
операции допускающие работу с большими скоростями

\end{enumerate}

\subarticle{Pro'sKit}

Отдельного обзора заслуживает инструмент и наборы
\href{http://www.proskit.com/}{Pro'sKit}
 / \href{http://www.proskit.msk.ru/index.html}{ru}:

\subsubarticle{Инструмент до 1000\,В}

Для электромонтажных работ обязательно приобретите комплект
высоковольтного инструмента до 1000\,В:

\begin{multicols}{2}
\noindent\includegraphics[width=\columnwidth]{00/fig/pros/PM-911.jpg}
\textbf{PM-911 Пассатижи 1\,кВ}

\noindent\includegraphics[width=\columnwidth]{00/fig/pros/PM-917.jpg}
\textbf{PM-917 Кусачки (бокорезы) 1\,кВ}
\end{multicols}

\subsubarticle{Хранение}

\begin{multicols}{2}

\noindent\includegraphics[width=\columnwidth]{00/fig/pros/103-132D.jpg}
\textbf{103-132D Кассетница для деталей и компонентов}

\noindent\includegraphics[width=\columnwidth]{00/fig/pros/SB-3428SB.jpg}
\textbf{SB-3428SB Портативная кассетница для саморезов и т.п.}
\end{multicols}

\subsubarticle{Радиомонтаж}

\begin{multicols}{2}

\noindent\includegraphics[width=\columnwidth]{00/fig/pros/8PK-30D.jpg}
\textbf{8PK-30D Кусачки миниатюрные}

\noindent\includegraphics[width=\columnwidth]{00/fig/pros/1PK-709.jpg}
\textbf{1PK-709 Длинногубцы-кусачки}

\noindent\includegraphics[width=\columnwidth]{00/fig/pros/1PK-055S.jpg}
\textbf{1PK-055S Длинногубцы изогнутые}

\noindent\includegraphics[width=\columnwidth]{00/fig/pros/1PK-29.jpg}
\textbf{1PK-29 Круглогубцы}

\noindent\includegraphics[width=\columnwidth]{00/fig/pros/1PK-101T.jpg}
\textbf{1PK-101T Пинцет прямой}

\noindent\includegraphics[width=\columnwidth]{00/fig/pros/1PK-3001E.jpg}
\textbf{1PK-3001E Клещи для зачистки проводов прецизионные (стриппер)}

\noindent\includegraphics[width=\columnwidth]{00/fig/pros/PD-374.jpg}
\textbf{PD-374 Тиски на струбцине}

\end{multicols}


\subsubarticle{Наборы}

\begin{multicols}{2}
\noindent\includegraphics[width=\columnwidth]{00/fig/pros/1PK-616B.jpg}
\textbf{1PK-616B Набор инструментов для электроники профессиональный}

\columnbreak

\noindent\includegraphics[width=\columnwidth]{00/fig/pros/1PK-813B.jpg}
\textbf{1PK-813B Набор базовых инструментов для электроники}
\end{multicols}

По личному опыту: в 1PK-813B не хватает мелкого мультиметра, стриппера
1PK-3001E, микрокусачек типа 8PK-30D, канифоли, ножа, настроечную отвертку
заменить индикаторной.

\subarticle{Прочие}

Попалась интересная недорогая отвертка:

\begin{multicols}{2}
\noindent\includegraphics[width=0.9\columnwidth]{00/fig/P1020966.jpg}

\noindent\includegraphics[width=0.9\columnwidth]{00/fig/P1020967.jpg}
\end{multicols}

Фиксация четкая, исполнение очень неплохое, позволяет добраться до узких мест.
Из минусов: ручка похоже не цельнометаллическая, при изломе есть риск 
распороть руку.

}{}


% % % \article{Технологии обучения}{Программы для подготовки учебных материалов}{
% % %
% % % Важным моментом является подготовка учебных материалов, чтобы любой желающий
% % % мог повторить ваш скратч.
% % %
% % % \subarticle{Виртуальные машины}
% % %
% % % Для подготовки материалов по программному обеспечению часто нужно выполнить
% % % действия, оказывающие сильное влияние на работоспосообность системы, или
% % % требующее предварительно подготовленного (начального) состояния.
% % %
% % % Некоторые операции типа установки операционной системы, не позволяют выполнить
% % % запись процесса, и принципиально требуют использования виртуальной машины, или
% % % аппаратной записи в выхода видеокарты.
% % %
% % % \subsubarticle{VMWare Player}
% % %
% % % Самый известный коммечрсекий продукт, в варианте Player раздается с сайта
% % % бесплатно для частного пользования.
% % %
% % % \subarticle{Подготовка текстовых материалов}
% % %
% % % \LaTeX\ требует некоторого обучения, и подходящего текстового редактора с
% % % подсветкой синтаксиса. Удобен очень богатыми возможностями по созданию
% % % макросов, реализацией различных указателей, и (профессиональным) вводом
% % % большого количества мат.формул.
% % %
% % % \subarticle{Скриншоты}
% % %
% % % \subsubarticle{GreenShot}
% % %
% % % \subarticle{Редактирование графики}
% % %
% % % \subsubarticle{GIMP}
% % %
% % % \subarticle{Видеозапись экрана}
% % %
% % % \subsubarticle{CamStudio}
% % %
% % % \subarticle{Видеомонтаж}
% % %
% % % \cp{http://rus-linux.net/MyLDP/mm/video-redaktory.html}
% % %
% % % \subsubarticle{Lightworks}
% % %
% % % Lightworks\ --- это высококачественный видеоредактор профессионального уровня, у
% % % которого недавно вышла бета-версия для Linux. Lightworks был возможно одной из
% % % первых компьютерных систем нелинейного монтажа, и разрабатывается с 1989 года. В
% % % мае 2010 года были аннонсированы версия с открытым исходным кодом и версии для
% % % Linux и Mac OS X. Бета-версия бесплатна для скачивания и использования, но
% % % дополнительные функции и расширенную поддержку кодеков разработчики предлагают
% % % за $60 в год.
% % %
% % % \menu{\url{http://www.lwks.com/}}
% % %
% % % \menu{\url{http://www.lwks.com/get-windows}}
% % %
% % % \includegraphics[]{}
% % %
% % % \subsubarticle{Cinelerra}
% % %
% % % Проф.пакет для видеомонтажа, Linux-only, сборок под win32 нет.
% % %
% % % }{Йа}
% % 
% % % \article{Технологии}{Готовим травильный аквариум}{
% % % 
% % % \subarticle{Введение}
% % % 
% % % \noindent\includegraphics[width=\columnwidth]{fig/00/smit/ee000008.png}
% % % 
% % % Данная разработка, как и любая другая у меня, делалась по принципу увидел клевую
% % % вещь\ --- захотел такую же. Сей проект\ --- это творчески доработанный аквариум
% % % для травления печатных плат в домашних условиях. Этих аквариумов уже гуляет
% % % солидное количество по сети.
% % % 
% % % Я хоть травлю платы не слишком часто, но иногда случается и так получается, что
% % % эти самые платы у меня плохо получаются. \smiley\ И вот однажды в поисках
% % % секрета получения хороших плат я наткнулся на
% % % \href{http://we.easyelectronics.ru/Tools/akvarium-dlya-travleniya-pechatnyh-plat.html}{вот
% % % эту штуку}.
% % % 
% % % Мне она понравилась и я захотел такую же. Но как и любая идея которая у меня
% % % зреет, эта ждала своего часа довольно долго. Пока в один прекрасный день я не
% % % уволился с работы, а до устройства на следующую у меня оставались три недели
% % % лишнего отпуска. Их-то я и решил потратить с пользой. Что из этого получилось я
% % % и хочу вам показать.
% % % 
% % % \subarticle{Выбор концепции и комплектующих или начинаем готовку}
% % % 
% % % Итак задача ясна и я начал думать как её осуществить. Тупо повторить разработку
% % % \textbf{JeckDigger}\ (ссылка выше) мне было неинтересно, ведь впереди целых три
% % % недели! И я начал думать, что же такого интересного можно сделать.
% % % С самим аквариум из оргстекла все ясно, тут что-то новое придумать сложно.
% % % Единственное, если у JeckDigger аквариум изготавливался из подручных средств, то
% % % у меня был доступ к фрезерному станку и я решил им воспользоваться по полной.
% % % Быстро наваял чертежи аквариума и мне их по дружбе сфрезировали. В качестве
% % % материала я использовал оргстекло толщиной 5 мм. Получилось как-то так :
% % % 
% % % \noindent\includegraphics[width=\columnwidth]{fig/00/smit/ee000017.png}
% % % \texbf{Чертеж аквариума}
% % % 
% % % \href{}{Файлы чертежей в формате .dwg}
% % % 
% % % А вот и деталировка:
% % % 
% % % \noindent\includegraphics[width=\columnwidth]{fig/00/smit/ee000018.png}
% % % \texbf{Основание}
% % % 
% % % Красным выделено углубление 2\,мм. Которое играет роль своеобразных
% % % направляющих. В них устанавливаются стенки аквариума. Сделал я их для лучшего
% % % позиционирования при слеивании аквариума ибо придавливать вертикальную стенку
% % % авквариума смазанную клеем к гладкой плоскости основание то еще удовольствие,
% % % добиться параллельности всех стенок очень сложно. А в таком варианте все удобно
% % % становится, да и клей в углубление хорошо заливается и не размазывается по всей
% % % поверхности.
% % % 
% % % \noindent\includegraphics[width=\columnwidth]{fig/00/smit/ee000019.png}
% % % \texbf{Фронтальная стенка}
% % % 
% % % Красным выделено углубление 2\,мм в виде ступеньки. Сделано оно опять же для
% % % удобства склеивания и для того, чтобы на выходе получить аккуратный
% % % параллелепипед аквариума. \smiley
% % % 
% % % \noindent\includegraphics[width=\columnwidth]{fig/00/smit/ee000020.png}
% % % \texbf{Боковая стенка}
% % % 
% % % Здесь никаких извращений. Обычный прямоугольник. \smiley
% % % \bigskip
% % % 
% % % Итак с самим аквариумом разобрались. Но этого конечно же мало для полного
% % % счастья. Как и у JeckDigger я решил добавить туда стандартный аквариумный
% % % распылитель.
% % % 
% % % \noindent\includegraphics[width=\columnwidth]{fig/00/smit/ee000021.png}
% % % \texbf{Аквариумный распылитель}
% % % 
% % % Распылитель служит своеобразной <<палкой-мешалкой>> травящего
% % % раствора. Ходит слух, что это помогает ускорить процесс травления. Скажу
% % % честно, я без понятия правда это или нет, но идея мне понравилась.
% % % 
% % % Для распылителя был необходим компрессор. Я выбрал самый
% % % миниатюрный какой только нашел\ --- АС-500.
% % % 
% % % \noindent\includegraphics[width=0.5\columnwidth]{fig/00/smit/ee000022.png}
% % % \texbf{Аквариумный компрессор}
% % % 
% % % Объем нашего аквариума небольшой поэтому такого компрессора вполне хватит.
% % % 
% % % Но если вы думаете, что это все, то вы плохо меня знаете. \smiley\ Разойдясь я
% % % уже не мог остановиться. Я решил прикрутить к всему этому добру нагреватель с
% % % температурной регулировкой. Опять же ходит слух, что в теплой воде процесс
% % % травления идет быстрее\footnote{\ а если объединить компрессор и нагреватель, то
% % % плата должна травиться вообще в две секунды \smiley\ }. Сказано\ --- сделано. С
% % % качестве нагревателя был использован обычный аквариумный нагреватель мощностью
% % % 75 Вт.
% % % 
% % % \noindent\includegraphics[width=\columnwidth]{fig/00/smit/ee000023.png}
% % % \texbf{Аквариумный нагреватель}
% % % 
% % % \emph{Примечание 2\ --- Не обращайте внимания, что на картинке указанна
% % % мощность 50\,Вт. Просто достойной фотографии нагревателя, который
% % % приобрел я сделать не удалось, поэтому в срочном порядке был подключен
% % % Google \smiley}.
% % % 
% % % Правда нагреватель пришлось доработать, но об этом речь пойдет дальше.
% % % 
% % % Итак, нагреватель есть, осталось прикрутить к нему терморегулятор. И тут
% % % появилась дилема: либо собирать схему самому либо купить уже готовый.
% % % После непродолжительных дебатов с самим собой решение было принято в пользу
% % % покупного. Ибо, делать плату самому было лень. Да и как её сделаешь, если
% % % аквариум для травления еще не готов? \smiley\ Налицо парадокс. \smiley
% % % 
% % % Свободного времени было хоть отбавляй, поэтому не откладывая дело в долгий ящик
% % % я пошел на радио рынок. После непродолжительных поисков я нашел, что искал:
% % % 
% % % \noindent\includegraphics[width=\columnwidth]{fig/00/smit/ee000024.png}
% % % \texbf{Плата регулятора температуры}
% % % 
% % % По параметром он конечно немного превосходил мои скромные запросы, но я был не в
% % % обиде.
% % % 
% % % Итак, вроде с большего было понятно что делать, оставалось обдумать детали.
% % % 
% % % Например, я захотел не только регулировать температуру кнопками, но и
% % % включать/выключать распылитель при помощи кнопки. Так же появилась небольшая
% % % проблема связанная с питанием. Заключалась она в том, что нагреватель и
% % % компрессор работают от напряжения $\sim$220\,В, плюс для питания остальной
% % % электроники планировалось использовать зарядное устройство для сотового
% % % телефона, а оно так же подключалось в сети $\sim$220\,В. А это означало, что для
% % % работы аквариума необходимо было занимать аж три розетки, что меня нисколько не
% % % устраивало. Поэтому я решил сделать, так чтобы для питания прибора можно было
% % % обойтись одним сетевым кабелем. Так же я решил всю электронику, включая
% % % компрессор запихнуть в деревянное основание, на которое уже устанавливать сам
% % % аквариум из оргстекла. Что же, концепция, с большего была ясна\ --- осталось
% % % дело за малым. \smiley
% % % 
% % % }{}
% % % %Алекс Смит {\href{mailto:zamuhrishka\@inbox.ru}}
% 
% 
\article{Хрюникс}{Собираем Cross Linux}{

В качествe примера применения возьмем относительно простое приложение:
многофункциональные настенные часы, с синхронизацией времени через \internet, с
будильником, медиапроигрывателем, блэкджеком и плюшками.

\subarticle{Linux для встраиваемых систем}

Linux для встраиваемых систем\footnote{\ будем называть его \emlinux}\ ---
популярный метод быстрого создания комплекса ПО для больших сложных приложений,
работающих на достаточно мощном железе, особенно предполагающих интенсивное
использование сетевых технологий.

За счет использования уже существующей и очень большой базы исходных текстов
ядра, библиотек и программ для \linux, бесплатно доступных в т.ч. и для
коммерческих приложений, можно на порядки сократить стоимость разработки
собственных программных компонентов, и при этом получить очень мощную команду
бесплатных стронних разработчиков, уже знакомых с созданием ПО для \linux.

Из недостатков можно отметить:
\begin{itemize}
  \item Отсутствие полноценной поддержки режима жесткого реального времени;
  \item Тяжелое ядро;
  \begin{itemize}
  \item Поддерживаются только мощные семейства процессоров;
  \item Значительные требования по объему \ram\ и общей производительности;
  \end{itemize}
  \item Дремучесть техспециалистов, контуженных ТурбоПаскалем и
Win\-dows\-ом;
\end{itemize}

Для \emph{сборки}\ \emlinux-системы используется метод кросс-компиляции, когда
используется \emph{кросс-тулчейн}, компилирующий весь комплект ПО для компьютера
с другой архитектурой\footnote{\ типичный пример\ --- сборка ПО на ПК c
процессором Intel i7 для Raspberry Pi или планшета на процессоре
AllWinner/Tegra/\ldots}.

\emlinux\ очень широко применяется на рынке мобильных устройств\footnote{\ в
т.ч. является основой Android}, и устройств интенсивно использующих сетевые
протоколы (роутеры, медиацентры).

\subarticle{Требования к системе сборки (\file{BUILD})}

Требования жесткие\ --- 4х-ядерный процессор, 4+\,Гб \ram, 64х-битный
дистрибутив \linux\ (рекомедую Debian), и никаких виртуалок.

Возможна установка системы на флешку, в этом случае требования к \ram\ еще более
ужесточаются\ --- потребуется каталоги с временными файлами смонтировать как
\file{tmpfs}.

Сборка под MinGW/Cygwin совершенно неживая. Если совсем никак без винды\ ---
используйте виртуалки, и будьте готовы ждать часами.

Можно попытаться сделать \file{билд-сервер}\ и на худшем железе, но будьте
готовы к тормозам или внезапному окончанию памяти\ --- ресурсоемка сборка
тяжелых библиотек типа \file{libQt} или крупных пакетов типа \file{gcc}.

В этом номере \Scr а описана сборка только базовой системы. Вы можете
попробовать поставить \linux\ на виртуалку, на флешку, и на жесткий диск (если
найдете место) и оценить возможности этих вариантов на сборке пакета \file{gcc}.

\subarticle{\file{Makefile}\ и пакеты}

Сборка выполняется утилитой \file{make}, описание структуры проекта для которой
прписано в файлах \file{Makefile} и \file{*.mk}.

В обычных дистрибутивах пакетами называют архивы скомпилированных программ,
устанавливаемых в \linux-систему. В нашем случае кросс-компиляции, будем
называть \term{пакетом}\ архив исходных текстов определенной программы или
компонента системы, вместе с секциеями мейкфайла, выполняющего ее компиляцию.

Также пакет может быть не связан с компиляцией ПО, а выполнять какую-то
вспомогательную работу.

Пакеты запускаются по своему имени вручную с помощью команды
\lstinputlisting{00_makesample.rc}

\subarticle{Структура системы \file{cross}}

Получите последнюю верию системы \file{cross}\ командой:

\lstinputlisting[title=git clone]{00_gitclone.rc}

\subsubarticle{скрипт генерации \file{Makefile}}

Главный \file{Makefile}\ генерируется по частям скриптом:

\lstinputlisting[title=mk.rc]{../cross/mk.rc}

Это было сделано для удобства вставки частей (.mk файлы) в документацию, в т.ч.
и в эту статью.

\subsubarticle{head.mk}

Задаются:

\begin{itemize}
  \item[\file{HW}] железка, на которой будет запускаться
  \item[\file{APP}] приложение 
\end{itemize}

В \file{config/hw/\$(HW).mk}\ прописываются перемнные:

\begin{itemize}
  \item[\file{ARCH}] архитектура железки
  \item[\file{CPU}] процессор и 
  \item[\file{TARGET}] \term{триплет целевой платформы} 
\end{itemize}

\lstinputlisting[title=head.mk]{../cross/head.mk}

\lstinputlisting[title=config/hw/qemu386.mk]{../cross/config/hw/qemu386.mk}

\lstinputlisting[title=config/arch/i386.mk]{../cross/config/arch/i386.mk}

\subsubarticle{каталоги, пакет \file{dirs}}

\begin{itemize}
  \item[\file{GZ}] зеркало архивов исходных текстов
  \item[\file{SRC}] сюда распаковываются исходные тексты
  \item[\file{TMP}] каталог для out-of-tree сборки 
  \item[\file{BUILD}] кросс-компилятор и другие программы \file{BUILD}-системы
  \item[\file{ROOT}] целевая файловая система \file{rootfs} (initrd)
  \item[\file{BOOT}] файлы, относящиеся к прцессу загрузки
\end{itemize}

Пакет \file{dirs}\ создает дерево каталогов. 

\lstinputlisting[title=dirs.mk]{../cross/dirs.mk}

\subsubarticle{версии пакетов}

В этой части прописаны версии пакетов.

\lstinputlisting[title=versions.mk]{../cross/versions.mk}

\subsubarticle{пакеты}

Здесь прописаны полные имена пакетов вместе с версиями. 

\lstinputlisting[title=packages.mk]{../cross/packages.mk}

\subsubarticle{комманды}

Макросы команд, используются далее:

\lstinputlisting[title=commands.mk]{../cross/commands.mk}

\subsubarticle{очистка \file{tmpfs}-каталогов}

Для каталогов, прописанных в \file{/etc/fstab}\ как \file{tmpfs}\ (файловая
система в \ram) для экономии ресурса флешки и ускорения сборки:

\lstinputlisting[title=/etc/fstab]{00_fstab.txt}

существует пакет \file{ramclean}, выполняющий их очистку.

Если вы используете такие каталоги, добавляете его вызов в конце вторым+ 
пакетом, чтобы освободить \ram\ для следующей сборки. 

\lstinputlisting[title=ramclean.mk]{../cross/ramclean.mk}

\subsubarticle{правила распаковки архивов исходниов}

\lstinputlisting[title=srcrules.mk]{../cross/srcrules.mk}

\subarticle{Сборка \file{BUILD}-части}

\subsubarticle{пакет \file{gz}}

Пакет \file{gz}\ выполняет загрузку из \internet а архивов исходных текстов
пакетов. 

Длительная и потребляющая трафик операция, нужен онлайн. По-хорошему архив
исходников тут было бы желательно загружать через пиринговые сети, а не
нагружать зеркала.

\lstinputlisting[title=gz.mk]{../cross/gz.mk}

\subsubarticle{набор перемнных для \file{configure}}

\begin{itemize}
  \item[\file{CFG}] общая часть запуска \file{configure}
  \item[\file{BCFG}] для \file{BUILD}-пакетов
  \item[\file{TCFG}] для \file{TAGET}-пакетов
\end{itemize}

\begin{itemize}
  \item{\file{--disable-nls}} диагностические сообщения и документация только на
английском
\item{\file{--prefix}} каталог установки скомпилированного пакета
\end{itemize}

\lstinputlisting[title=cfg.mk]{../cross/cfg.mk}

\subsubarticle{binutils}

Пакет \file{binutils}\ включает ассемблер, линкер и вспомогательные программы
для работы с объектными файлами в формате \term{ELF}. 

\begin{itemize}
  \item{\file{--target}} триплет целевой системы 
  \item{\file{--with-sysroot}} каталог с include/lib файлами целевой системы
  \item{\file{--with-native-system-header-dir}} относительно \file{SYSROOT}
\end{itemize}

\lstinputlisting[title=binutils.mk]{../cross/binutils.mk}

}{}


\end{document}
