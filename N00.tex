\documentclass[columnlines]{papertex}

% \documentclass[oneside]{papertex}%{book}
% %http://www.latex-community.org/forum/viewtopic.php?f=28&t=851
% %http://www.ctan.org/tex-archive/macros/latex/contrib/papertex
% %http://nachollopis.com/papertex/example.pdf
% 
% 
% % paper layout for screen reading
% %\usepackage[paperwidth=15cm,paperheight=10cm,margin=5mm]{geometry} % 
 
% pdflatex options
\usepackage[unicode]{hyperref}
\newcommand{\email}[1]{$<$\href{mailto:#1}{#1}$>$}

% i18n and encodings
\usepackage[T1,T2A]{fontenc}
\usepackage[utf8]{inputenc}
\usepackage[english,russian]{babel}
\usepackage{indentfirst}

% ifthen package used for extra hints later
\usepackage{ifthen}

\newcommand{\setdoclang}[1]{\def\doclang{#1}}
\setdoclang{en}

\newcommand{\en}[1]{\ifthenelse{\equal{\doclang}{en}}{#1}{}}
\newcommand{\ru}[1]{\ifthenelse{\equal{\doclang}{ru}}{#1}{}}

% \newcommand{\titleen}[1]{\ifthenelse{\equal{\doclang}{en}}{\title{#1}}{}}
% \newcommand{\titleru}[1]{\ifthenelse{\equal{\doclang}{ru}}{\title{#1}}{}}
% 
% % relative sectioning
% \newcounter{relsecount}\setcounter{relsecount}{0}
% \newcommand{\secup}{\addtocounter{relsecount}{1}}
% \newcommand{\secdown}{\addtocounter{relsecount}{-1}}
% \newcommand{\secrel}[1]{
% 	\ifthenelse{\equal{\value{relsecount}}{0}}{\chapter{#1}}{}
% 	\ifthenelse{\equal{\value{relsecount}}{-1}}{\section{#1}}{}
% 	\ifthenelse{\equal{\value{relsecount}}{-2}}{\subsection{#1}}{}
% 	\ifthenelse{\equal{\value{relsecount}}{-3}}{\subsubsection{#1}}{}
% }
% \newcommand{\secen}[1]{\ifthenelse{\equal{\doclang}{en}}{\secrel{#1}}{}}
% \newcommand{\secru}[1]{\ifthenelse{\equal{\doclang}{ru}}{\secrel{#1}}{}}

% misc
% \newcommand{\cp}[1]{\footnote{\ \copyright\url{#1}}}

\usepackage{wasysym} % smileys

\renewcommand{\logo}{\mylogo}

% shorts

\newcommand{\scr}{scr\en{atcher}\ru{этчер}}

\newcommand{\internet}{\en{Internet}\ru{Интернет}} 

% authors

\newcommand{\authorPonyatov}{\authorandplace{\en{Dmitry
Ponyatov}\ru{Дмитрий Понятов}}{\email{dponyatov@gmail.com}}}


\setdoclang{ru}

\newcommand{\scrT}{\en{Scratcher}\ru{Скрэтчер}\ \#00}
\edition{\scrT}

\begin{document}

\begin{frontpage}

\firstimage{logo/chbz.png}{}

\firstnews{\scrT}{Online журнал для \scr ов\ --- людей, чье хобби создавать
вещи и технологии по следам уже существующих, в сотый раз изобретать велосипед,
чтобы разобраться как оно работает, научиться делать самому, а возможно найти
новый или забытый способ что-то сделать, и конечно получить удовольствие от
процесса поиска.}{}
\secondnews{Персональное производство}{}{}{}{}{}

\begin{indexblock}{\ru{В выпуске}}
\indexitem{Об этом журнале}{1}
\indexitem{Как сказать "Начать с нуля"\ ?}{2}
\indexitem{Персональное производство\ --- еще один шаг к реконизму}{3}
\indexitem{Злобный янки в 3D-танке}{4}
\indexitem{Принципы скрэтчера}{princ}

\end{indexblock}

\begin{authorblock}
\textbf{\en{Editors:}\ru{Редакция:}}\\
\email{dponyatov@gmail.com}\\
{\small\url{https://github.com/ponyatov/scratcher}}
\end{authorblock}

\end{frontpage}

\newsection{От редакции}

\begin{editorial}{1}
	{Об этом журнале}
	{Сделай сам, расскажи другим}
	{1}

Наблюдая современные информационные тренды в \internet е, можно заметить, что
большое внимание уделяется различным самоделкам, DIY, 3D-принтерам, любительской
электронике и концептам различных гаджетов.

Если попробовать взглянуть немного дальше, можно заметить все более и более
заметное развитие такого явления как <<Персональное производство>>\ --- большой
интерес вызывает возможность создания и изготовления уникальных вещей, 
нужных только конкретному человеку.

С другой сторны, все усложняющиеся вещи и технологии вызывают у людей желание
начать с нуля, создать что-то пользуясь старыми приемами. В клинических
случаях попадаются особи, испытывающие дикий баттхерт от глобализации и
массового производства, и бегущие подальше от цивилизации, прихватив с собой
генератор и мобильник с \internet ом \smiley.

\emph{
Вполне можно ожидать, что эти тенденции приведут к появлению и оформлению нового
культурного течения, которое можно назвать} \textbf{скрэтчинг}\footnote{\ тут бы
хорошо подошло слово <<рукоблудие>>, но термин к сожалению уже занят, а
другого русского аналога подобрать пока не удалось}.

Этот журнал создан для \scr ов\ --- людей, чье хобби создавать вещи и технологии
по следам уже существующих, в сотый раз изобретать велосипед, чтобы разобраться
как оно работает, научиться делать самому, а возможно найти новый или забытый
способ что-то сделать, и конечно получить удовольствие от процесса поиска. 
	
\authorPonyatov
\end{editorial}

\begin{news}{2}
	{Как сказать "Начать с нуля"\ ?}
	{}
	{КопиПаста}
	{2}
	
\begin{itemize}
  \item
На английском — to start from scratch; to start over.
  \item
На испанском — empezar de cero; empezar de nuevo.
  \item
На итальянском — partire dal niente.
\end{itemize}

В общем-то во всех языках мы видим <<кальку>>, выделяется только одно,
содержащее слово <<scratch>> (царапина, черта).

С момента его возникновения, это выражение немного поменяло свое значение.
Сейчас оно используется, когда мы хотим сказать <<начать снова, начать с
начала>> в том смысле, что мы потерпели поражение при первой попытке.

Фраза родилась в конце 19-го века и тогда просто значила <<начинать без
преимуществ>>. Слово <<scratch>> использовалось с 18-го века как спортивный
термин, обозначающий линию старта, прочерченную на земле. Впервые такая линия
упоминалась в описании игры в крикет\ --- на ней стоял игрок, отбивающий мяч.

<<Start from scratch>> в качестве понятия <<начинать с нуля>> пришло к нам из
бокса. Прочерченная линия определяла позиции боксеров, когда они стояли друг
напротив друга в начале поединка. Отсюда также произошло выражение <<up to
scratch>>, (быть на должной высоте, в прекрасной форме), т.е. соответствовать
стандартам, предъявляемым боксерам, делающим заявку на матч.

Позднее <<scratch>> стали называть любую стартовую точку в бегах. Термин стали
использовать в <<гандикап>>-соревнованиях (handicap), в которых более слабый
участник получает фору. Например, в велоспорте те, у кого нет преимуществ, стоят
на линии, в то время как остальные стоят впереди. Другие виды спорта, особенно
гольф, заимствовали переносное значение <<scratch>> как термин для обозначения
<<без преимуществ\ --- начинать с нуля>>.

В The Fort Wayne Gazette (апрель 1887) содержится самое раннее упоминание <<start
from scratch>> — в репортаже о <<‘no-handicap>> велосипедной гонке:

<<It was no handicap. Every man was qualified to and did start from scratch.>>

По моим наблюдениям, <<start from scratch>> употребляется чаще в письменной речи
(например, уже несколько раз видела его в статьях в интернете), а <<to start
over>>\ --- в разговорной (слышала в американском сериале).

\authorandplace{\copyright\ Юлия Горбунова}{\href{http://www.lingvaroom.ru/kak-skazat-nachat-s-nulya/}{оригинал}}
	
\end{news}

\begin{news}{2}
	{Персональное производство}
	{еще один шаг к реконизму}
	{КопиПаста}
	{3}

Один из важных моментов в построении реконистической экономики — это
трансформация традиционного, корпоративного производства, основанного на
обязательной организации, как в смысле объединения людей, средств производства,
финансовых и материальных ресурсов, так и в смысле появления так называемых
юридических лиц как практически единственных субъектов производства. Такое
производство в значительной части сфер деятельности будет вытесняться
индивидуальным производством, когда  любой желающий, используя так называемые
микрофабрики — миниатюрный комплект универсального оборудования, сможет
производить достаточно широкую линейку продукции, как для личного пользования,
так и для продажи. Произойдет нечто вроде возврата к ремесленному производству
средневековья и даже к натуральному хозяйству, но на неизмеримо более высоком
технологическом уровне. Особую ценность в таких условиях обретет информация —
продаваться будет не товар, а инструкция для микрофабрики, как данный товар
изготовить. Конечно, такие инструкции будет не только продаваться, но и
распространяться бесплатно, а также вороваться. Разумеется, это серьезно
поменяет привычную нам социально-экономическую систему.

В последнем номере журнала «Наука и жизнь» (№8 за 2012 год), появилась небольшая
заметка, в которой рассказывается о разработке профессора Массачусетского
технологического института Нила Гершенфельда, который предложил концепцию
миниатюрной фабрики-лаборатории (Fab Lab). Фабрика-лаборатория представляет
собой комплекс станков, совместно работающих под управлением персонального
компьютера. Идея Гершенфельда получила широкое распространение и десятки
университетов и исследовательских центров экспериментируют с такими
мини-фабриками. В России первая такая фабрика создана в Московском институте
стали сплавов, в ее составе фрезерный станок для обработки древесины, пластиков
и мягких металлов, гравировальный прецизионный станок для производства печатных
плат, установка лазерной резки, плоттер для раскроя гибких материалов и
производства гибких микросхем, и 3D-принтер, предназначенный для изготовления
любых изделий из ABS-пластика.

Так что, возможно, что лет через десять, для того чтобы поменять надоевший
мобильный телефон, мы будем заходить на сайт какой-нибудь Нокии, скачивать файл
с данными, запускать его в программе на домашнем компьютере, а стоящий на
тумбочке агрегат, очертаниями смахивающий на современное МФУ, погудев пару
минут, выбросит в приемный лоток еще горячую, пахнущую свежим пластиком
мобилку\ldots
\smiley
	
\authorandplace{\copyright\ AG}{\href{http://blog42.ws/personalnoe-proizvodstvo-eshhe-odin-shag-k-rekonizmu/}{оригинал}}	
\end{news}

\newsection{Новости технологий}

\begin{news}{2}
	{Злобный янки в 3D-танке}
	{Полевые 3D-принтеры на службе американской армии}
	{Новости технологий}
	{4}

Пока специалисты в области 3D-печати рассуждают о перспективах приенения
технологии, а энтузиасты осторожно говорят о потенциальной возможности печати
необходимого скарба сразу на лунной базе (чтобы не тащить лишнее с Земли),
американская армия без всяких промедлений нашла применение 3D-печати уже сейчас.
Военные США стали использовать мобильные лаборатории Expeditionary Lab Mobile с
3D-принтерами в комплекте.

\image{fig/00/01.jpg}{Чебураторы на тропе войны}

Основными задачами лабораторий Expeditionary Lab Mobile (сокращённо — ELM) будет
изготовление одноразовых инструментов для нужд армии, а также внесение
корректирующих дополнений в уже существующее оборудование — «полевое»
использование часто требует определённой доводки. В качестве примера приводится
случай, когда войска получают партию карманных фонарей с дефектом – быстро
выходящим из строя предохранителем выключателя. Находясь в кармане у военного,
такой фонарь может самопроизвольно включиться и либо выдать местонахождение
бойца, либо впустую разрядить батарейки. Однако, имея под рукой ELM, можно
быстро допечатать предохранители, без необходимости отсылки всей партии обратно
в США для замены. 

\image{fig/00/03.jpg}{}

Ещё одним примером можно назвать реальный случай недоработки в
конструкции миноискателя, приведший к тому, что время работы прибора из-за
иракской жары сократилось с восьми часов до 45 минут. В результате во время
многодневных миссий солдаты были вынуждены носить большое количество
дополнительных батарей. Использование ELM позволило сконструировать адаптер для
использования батарей другого типа и увеличить время работы миноискателя до
девяти часов.

Expeditionary Lab Mobile представляет собой стандартный грузовой контейнер
(6,1$\times$2,4 м), внутри которого находятся 3D-принтер, специальные станки с
ЧПУ (для изготовления более сложных деталей из стали и алюминия) и набор
традиционных инструментов: резак, сварочный аппарат, циркулярная пила,
маршрутизатор, лобзик и сабельная пила. Кроме того, в комплекте ELM имеется
спутниковое оборудование связи для проведения телеконференций с чиновниками и
инженерами в США – для оперативных корректировок работы. При каждой лаборатории
будут находиться два инженера. Все лаборатории будут связаны между собой единой
компьютерной сетью.

\image{fig/00/02.jpg}{}

Стоит отметить, что подобный способ изготовления износившихся или недостающих
деталей довольно дорог: стоимость каждой лаборатории составляет около 2,8
миллиона долларов. Планируется, что первые ELM будут испытаны в Афганистане.
Кроме того, можно надеяться, что успешное применение новых технологий на <<поле
боя>> будет способствовать их внедрению для мирных операций. Например, во время
стихийных бедствий.

\authorandplace{\copyright\ Компьютерра, Николай
Маслухин}{\href{http://www.computerra.ru/50860/polevyie-3d-printeryi-na-sluzhbe-amerikans/}{оригинал}}
\end{news}

\begin{editorial}{1}
	{Принципы скрэтчера}
	{Чем оно отличается от прочего DIY}
	{princ}
	
\begin{itemize}
  
\item Из говна и палок
  
Чем больше г и кривее палки, тем круче \scr
  
\item Сделай сам, расскажи другим
  
Необходим активный обмен информацией для мимимизации и так больших расходов на
избретение колес
  
\item Минимум покупных изделий

В идеале изготовление всего из чисто природных материлов и без стартового
инструмента

\item Все покупные ништяки должны быть всегда доступны в любом ближайшем
магазине

Чтобы каждый мог легко и быстро повторить понравившийся хак.

\emph{Следует обратить внимание, что этому принципу противоречит использование
техно-мусора, различных деталей от старой стиралки и т.п.\ --- вот сколько
сейчас у вас например сломанных стиралок, или дохлых телевизоров в доме ?}

\emph{Еще одно противоречие\ --- покупка комплектухи по почте в Китае, и заказ
редких компонентов в магазинах}

\item Покупаться должны \textbf{самые дешевые}\ и самые кривые
комплектующие

Но при этом не нужно скатываться на использование раритета\ --- см. доступность.

\item Приоритетно использование более ранней ступени
\term{технологического передела}

Например вместо использование готового заводского сверла взять хвостовик от
сломанного, и выпилить сверло самому. Правильнее было бы взять твердосплавную
заготовку, но это противоречит принципу доступности, т.к. их нет в
доступных магазинах. Вариант использование куска проката из инструментального
сплава лучше, потому что можно еще повыделываться с термичкой \smiley

\item Должно использоваться \term{открытое программное обеспечение}

Причем написанное целиком самостоятельно на ассемблере, ну или хотя бы собрать
\href{http://cross-lfs.org/}{Cross Linux From Scratch}\ для DIY компьютера,
спаянного из отдельных деталей с помощью самодельного паяльника.

В процессе неплохо попутно изобрести пару уникальных языков программирования,
написать на них операционную систему и комплект программного обеспечения.

\item Желательно использовать нетиповые приемы работы и технологии

\item При разработке конструкций нужно стремиться использовать малоизвестные и
уникальные конструктивные решения

\item Максимум самодельного инструмента

\item Идеал \scr а\ --- пройти всю технологическую цепочку от каменного 
рубила до обрабатывающего центра с ЧПУ

И с разгона заскочить еще дальше, обогнав текущие лабораторные разработки по
3D-печати, зональной плавке и прочим свежакам технологии

\item Минимум повторов готовых изделий и унификации

Каждая поделка должна быть прекрасна в своей уникальности, и ее область
применения должна быть максимально узкозаточенной под ваши задачи. Применение
унификации, общеизвестных конструктивных решений и принципов работы неприемлемо,
т.к. какой смысл повторять уже готовое изделие, которое можно купить ?!

\item Больше науки

Копайте книги по математике, физике и химии, больше статей и техрасчетов.
Чем больше матана и самопала, тем выше левел. Не забывайте про пропагандизм
достижений на форумах (особенно нетематических) и в оффлайне.

\item Больше синей изоленты

\item Обязательно используйте ардуину

Даже если устройство вообще не предполагает использование электричества\ ---
прикрутите микроконтроллер изолентой, и подключите к нему компьютер.

\item На фото и видео обязателен ковер  

Для успеха проекта обязательно нужен ковер

\item На демонстрационном видео должно что-нибудь отвалиться или чпохнуть
волшебным синим дымом
   
\end{itemize}	

\end{editorial}

\begin{editorial}{1}
	{}
	{}
	{onprinc}
	
Набор принципов \scr а выглядит похоже на инструкцию <<Как просрать полимеры>>,
поэтому как и в любом другом деле, не нужно доводить их исполнение до фанатизма.
\emph{Новизна и уникальность} на первом месте, и не надо забывать что это все же
хобби, а не жизненная миссия. Нужно всего лишь следить за соблюдением баланса
между потраченными средствами, временем, и полученным от процесса удовольствием.

Главное достоинство отработанных вещей и технологий\ --- на их доводку и
проверку уже было потрачено гиганское количество ресурсов. Самодельные аналоги в
любом случае будут хуже и на порядки дороже, чем серийное изделие, за редким
исключением узконишевого использования, для которого готовое решение почему-то
не подходит.

Из положительных эффектов \scr ства можно отметить хорошие общетехнические
знания, и умение при необходимости быстро слепить <<костыль>> (временное решение
проблемы) из подручных ресурсов.
	
\end{editorial}

\end{document}
