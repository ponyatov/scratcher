
\article{Хрюникс}{Собираем Cross Linux}{

В качествe примера применения возьмем относительно простое приложение:
многофункциональные настенные часы, с синхронизацией времени через \internet, с
будильником, медиапроигрывателем, блэкджеком и плюшками.

\subarticle{Linux для встраиваемых систем}

Linux для встраиваемых систем\footnote{\ будем называть его \emlinux}\ ---
популярный метод быстрого создания комплекса ПО для больших сложных приложений,
работающих на достаточно мощном железе, особенно предполагающих интенсивное
использование сетевых технологий.

За счет использования уже существующей и очень большой базы исходных текстов
ядра, библиотек и программ для \linux, бесплатно доступных в т.ч. и для
коммерческих приложений, можно на порядки сократить стоимость разработки
собственных программных компонентов, и при этом получить очень мощную команду
бесплатных стронних разработчиков, уже знакомых с созданием ПО для \linux.

Из недостатков можно отметить:
\begin{itemize}
  \item Отсутствие полноценной поддержки режима жесткого реального времени;
  \item Тяжелое ядро;
  \begin{itemize}
  \item Поддерживаются только мощные семейства процессоров;
  \item Значительные требования по объему \ram\ и общей производительности;
  \end{itemize}
  \item Дремучесть техспециалистов, контуженных ТурбоПаскалем и
Win\-dows\-ом;
\end{itemize}

Для \emph{сборки}\ \emlinux-системы используется метод кросс-компиляции, когда
используется \emph{кросс-тулчейн}, компилирующий весь комплект ПО для компьютера
с другой архитектурой\footnote{\ типичный пример\ --- сборка ПО на ПК c
процессором Intel i7 для Raspberry Pi или планшета на процессоре
AllWinner/Tegra/\ldots}.

\emlinux\ очень широко применяется на рынке мобильных устройств\footnote{\ в
т.ч. является основой Android}, и устройств интенсивно использующих сетевые
протоколы (роутеры, медиацентры).

\subarticle{Требования к системе сборки (\file{BUILD})}

Требования жесткие\ --- 4х-ядерный процессор, 4+\,Гб \ram, 64х-битный
дистрибутив \linux\ (рекомедую Debian), и никаких виртуалок.

Возможна установка системы на флешку, в этом случае требования к \ram\ еще более
ужесточаются\ --- потребуется каталоги с временными файлами смонтировать как
\file{tmpfs}.

Сборка под MinGW/Cygwin совершенно неживая. Если совсем никак без винды\ ---
используйте виртуалки, и будьте готовы ждать часами.

Можно попытаться сделать \file{билд-сервер}\ и на худшем железе, но будьте
готовы к тормозам или внезапному окончанию памяти\ --- ресурсоемка сборка
тяжелых библиотек типа \file{libQt} или крупных пакетов типа \file{gcc}.

В этом номере \Scr а описана сборка только базовой системы. Вы можете
попробовать поставить \linux\ на виртуалку, на флешку, и на жесткий диск (если
найдете место) и оценить возможности этих вариантов на сборке пакета \file{gcc}.

\subarticle{\file{Makefile}\ и пакеты}

Сборка выполняется утилитой \file{make}, описание структуры проекта для которой
прписано в файлах \file{Makefile} и \file{*.mk}.

В обычных дистрибутивах пакетами называют архивы скомпилированных программ,
устанавливаемых в \linux-систему. В нашем случае кросс-компиляции, будем
называть \term{пакетом}\ архив исходных текстов определенной программы или
компонента системы, вместе с секциеями мейкфайла, выполняющего ее компиляцию.

Также пакет может быть не связан с компиляцией ПО, а выполнять какую-то
вспомогательную работу.

Пакеты запускаются по своему имени вручную с помощью команды
\lstinputlisting{00_makesample.rc}

\subarticle{Структура системы \file{cross}}

Получите последнюю верию системы \file{cross}\ командой:

\lstinputlisting[title=git clone]{00_gitclone.rc}

\subsubarticle{скрипт генерации \file{Makefile}}

Главный \file{Makefile}\ генерируется по частям скриптом:

\lstinputlisting[title=mk.rc]{../cross/mk.rc}

Это было сделано для удобства вставки частей (.mk файлы) в документацию, в т.ч.
и в эту статью.

\subsubarticle{head.mk}

Задаются:

\begin{itemize}
  \item[\file{HW}] железка, на которой будет запускаться
  \item[\file{APP}] приложение 
\end{itemize}

В \file{config/hw/\$(HW).mk}\ прописываются перемнные:

\begin{itemize}
  \item[\file{ARCH}] архитектура железки
  \item[\file{CPU}] процессор и 
  \item[\file{TARGET}] \term{триплет целевой платформы} 
\end{itemize}

\lstinputlisting[title=head.mk]{../cross/head.mk}

\lstinputlisting[title=config/hw/qemu386.mk]{../cross/config/hw/qemu386.mk}

\lstinputlisting[title=config/arch/i386.mk]{../cross/config/arch/i386.mk}

\subsubarticle{каталоги, пакет \file{dirs}}

\begin{itemize}
  \item[\file{GZ}] зеркало архивов исходных текстов
  \item[\file{SRC}] сюда распаковываются исходные тексты
  \item[\file{TMP}] каталог для out-of-tree сборки 
  \item[\file{BUILD}] кросс-компилятор и другие программы \file{BUILD}-системы
  \item[\file{ROOT}] целевая файловая система \file{rootfs} (initrd)
  \item[\file{BOOT}] файлы, относящиеся к прцессу загрузки
\end{itemize}

Пакет \file{dirs}\ создает дерево каталогов. 

\lstinputlisting[title=dirs.mk]{../cross/dirs.mk}

\subsubarticle{версии пакетов}

В этой части прописаны версии пакетов.

\lstinputlisting[title=versions.mk]{../cross/versions.mk}

\subsubarticle{пакеты}

Здесь прописаны полные имена пакетов вместе с версиями. 

\lstinputlisting[title=packages.mk]{../cross/packages.mk}

\subsubarticle{комманды}

Макросы команд, используются далее:

\lstinputlisting[title=commands.mk]{../cross/commands.mk}

\subsubarticle{очистка \file{tmpfs}-каталогов}

Для каталогов, прописанных в \file{/etc/fstab}\ как \file{tmpfs}\ (файловая
система в \ram) для экономии ресурса флешки и ускорения сборки:

\lstinputlisting[title=/etc/fstab]{00_fstab.txt}

существует пакет \file{ramclean}, выполняющий их очистку.

Если вы используете такие каталоги, добавляете его вызов в конце вторым+ 
пакетом, чтобы освободить \ram\ для следующей сборки. 

\lstinputlisting[title=ramclean.mk]{../cross/ramclean.mk}

\subsubarticle{правила распаковки архивов исходниов}

\lstinputlisting[title=srcrules.mk]{../cross/srcrules.mk}

\subarticle{Сборка \file{BUILD}-части}

\subsubarticle{пакет \file{gz}}

Пакет \file{gz}\ выполняет загрузку из \internet а архивов исходных текстов
пакетов. 

Длительная и потребляющая трафик операция, нужен онлайн. По-хорошему архив
исходников тут было бы желательно загружать через пиринговые сети, а не
нагружать зеркала.

\lstinputlisting[title=gz.mk]{../cross/gz.mk}

\subsubarticle{набор перемнных для \file{configure}}

\begin{itemize}
  \item[\file{CFG}] общая часть запуска \file{configure}
  \item[\file{BCFG}] для \file{BUILD}-пакетов
  \item[\file{TCFG}] для \file{TAGET}-пакетов
\end{itemize}

\begin{itemize}
  \item{\file{--disable-nls}} диагностические сообщения и документация только на
английском
\item{\file{--prefix}} каталог установки скомпилированного пакета
\end{itemize}

\lstinputlisting[title=cfg.mk]{../cross/cfg.mk}

\subsubarticle{binutils}

Пакет \file{binutils}\ включает ассемблер, линкер и вспомогательные программы
для работы с объектными файлами в формате \term{ELF}. 

\begin{itemize}
  \item{\file{--target}} триплет целевой системы 
  \item{\file{--with-sysroot}} каталог с include/lib файлами целевой системы
  \item{\file{--with-native-system-header-dir}} относительно \file{SYSROOT}
\end{itemize}

\lstinputlisting[title=binutils.mk]{../cross/binutils.mk}

}{}
