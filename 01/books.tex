\article{Школотрон}{Обзор книг}{

\subarticle{\cite{svoren} Сворень}

\subarticle{\cite{bermant} Бермант А.Ф, Араманович И.Г.\\
Краткий курс математического анализа для ВТУЗов}



\subarticle{\cite{ltspice} Володин В.Я.\\
LTspice: компьютерное моделирование электронных схем}

Руководство для эффективного освоения бесплатного SPICE-симулятора LTspice,,
предназначенного для компьютерного моделирования электронных схем, является
наиболее полным описанием программы, пользующейся заслуженной популярно-
стью как среди любителей, так и среди профессионалов. Содержит рекомендации,
позволяющие быстро начать работать с симулятором, и в то же время включает пол-
ное описание интерфейса, библиотеки схемных элементов и директив моделирова-
ния. Рассматриваются процесс настройки схемных элементов, связь текстового опи-
сания схемных элементов с графическим интерфейсом программы, редактор схем,,
редактор символов и плоттера. Подробно описаны вопросы создания и тестирования
нелинейных индуктивностей и трансформаторов, вызывающие наибольшие затруд-
нения у начинающих. Большое внимание уделено процессу адаптации сторонних
моделей, а также созданию собственных моделей схемных компонентов. Приводится
методика моделирования электромагнитных компонентов с разветвленным сердеч-
ником. Изложение сопровождается большим количеством практических примеров и
иллюстраций, облегчающих усвоение сложного материала. Прилагаемый DVD со-
держит видеоуроки для освоения симулятора, примеры из книги и авторскую биб-
лиотеку моделей популярных ШИМ-контроллеров.

}{}
