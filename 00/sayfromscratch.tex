\article{КопиПаста}{Как сказать "Начать с нуля"\,?}{

\begin{itemize}
  \item
На английском — to start from scratch; to start over.
  \item
На испанском — empezar de cero; empezar de nuevo.
  \item
На итальянском — partire dal niente.
\end{itemize}

В общем-то во всех языках мы видим <<кальку>>, выделяется только одно,
содержащее слово <<scratch>> (царапина, черта).

С момента его возникновения, это выражение немного поменяло свое значение.
Сейчас оно используется, когда мы хотим сказать <<начать снова, начать с
начала>> в том смысле, что мы потерпели поражение при первой попытке.

Фраза родилась в конце 19-го века и тогда просто значила <<начинать без
преимуществ>>. Слово <<scratch>> использовалось с 18-го века как спортивный
термин, обозначающий линию старта, прочерченную на земле. Впервые такая линия
упоминалась в описании игры в крикет\ --- на ней стоял игрок, отбивающий мяч.

<<Start from scratch>> в качестве понятия <<начинать с нуля>> пришло к нам из
бокса. Прочерченная линия определяла позиции боксеров, когда они стояли друг
напротив друга в начале поединка. Отсюда также произошло выражение <<up to
scratch>>, (быть на должной высоте, в прекрасной форме), т.е. соответствовать
стандартам, предъявляемым боксерам, делающим заявку на матч.

Позднее <<scratch>> стали называть любую стартовую точку в бегах. Термин стали
использовать в <<гандикап>>-соревнованиях (handicap), в которых более слабый
участник получает фору. Например, в велоспорте те, у кого нет преимуществ, стоят
на линии, в то время как остальные стоят впереди. Другие виды спорта, особенно
гольф, заимствовали переносное значение <<scratch>> как термин для обозначения
<<без преимуществ\ --- начинать с нуля>>.

В The Fort Wayne Gazette (апрель 1887) содержится самое раннее упоминание
<<start from scratch>> — в репортаже о <<‘no-handicap>> велосипедной гонке:

<<It was no handicap. Every man was qualified to and did start from scratch.>>

По моим наблюдениям, <<start from scratch>> употребляется чаще в письменной речи
(например, уже несколько раз видела его в статьях в интернете), а <<to start
over>>\ --- в разговорной (слышала в американском сериале).

}{\copyright\ \href{http://www.lingvaroom.ru/kak-skazat-nachat-s-nulya/}{Юлия
Горбунова}}
