
\article{КопиПаста}{Персональное производство\\еще один шаг к реконизму}{

Один из важных моментов в построении реконистической экономики — это
трансформация традиционного, корпоративного производства, основанного на
обязательной организации, как в смысле объединения людей, средств производства,
финансовых и материальных ресурсов, так и в смысле появления так называемых
юридических лиц как практически единственных субъектов производства. Такое
производство в значительной части сфер деятельности будет вытесняться
индивидуальным производством, когда  любой желающий, используя так называемые
микрофабрики — миниатюрный комплект универсального оборудования, сможет
производить достаточно широкую линейку продукции, как для личного пользования,
так и для продажи. Произойдет нечто вроде возврата к ремесленному производству
средневековья и даже к натуральному хозяйству, но на неизмеримо более высоком
технологическом уровне. Особую ценность в таких условиях обретет информация —
продаваться будет не товар, а инструкция для микрофабрики, как данный товар
изготовить. Конечно, такие инструкции будет не только продаваться, но и
распространяться бесплатно, а также вороваться. Разумеется, это серьезно
поменяет привычную нам социально-экономическую систему.

В последнем номере журнала «Наука и жизнь» (№8 за 2012 год), появилась небольшая
заметка, в которой рассказывается о разработке профессора Массачусетского
технологического института Нила Гершенфельда, который предложил концепцию
миниатюрной фабрики-лаборатории (Fab Lab). Фабрика-ла\-бо\-ра\-то\-рия
представляет собой комплекс станков, совместно работающих под управлением
персонального компьютера. Идея Гершенфельда получила широкое распространение и
десятки университетов и исследовательских центров экспериментируют с такими
мини-фабриками. В России первая такая фабрика создана в Московском институте
стали сплавов, в ее составе фрезерный станок для обработки древесины, пластиков
и мягких металлов, гравировальный прецизионный станок для производства печатных
плат, установка лазерной резки, плоттер для раскроя гибких материалов и
производства гибких микросхем, и 3D-принтер, предназначенный для изготовления
любых изделий из ABS-пластика.

Так что, возможно, что лет через десять, для того чтобы поменять надоевший
мобильный телефон, мы будем заходить на сайт какой-нибудь Нокии, скачивать файл
с данными, запускать его в программе на домашнем компьютере, а стоящий на
тумбочке агрегат, очертаниями смахивающий на современное МФУ, погудев пару
минут, выбросит в приемный лоток еще горячую, пахнущую свежим пластиком
мобилку\ldots
\smiley

}{\copyright\
\href{http://blog42.ws/personalnoe-proizvodstvo-eshhe-odin-shag-k-rekonizmu/}{AG}}

