
\article{Об этом журнале}{Сделай сам, расскажи другим}{

Наблюдая современные информационные тренды в \internet е, можно заметить, что
большое внимание уделяется различным самоделкам, DIY, 3D-прин\-те\-рам,
любительской электронике и концептам различных гаджетов.

Если попробовать взглянуть немного дальше, можно заметить все более и более
заметное развитие такого явления как <<Персональное производство>>\ --- большой
интерес вызывает возможность создания и изготовления уникальных вещей, нужных
только конкретному человеку.

С другой сторны, все усложняющиеся вещи и технологии вызывают у людей желание
начать с нуля, создать что-то пользуясь старыми приемами. В клинических случаях
попадаются особи, испытывающие дикий баттхерт от глобализации и массового
производства, и бегущие подальше от цивилизации, прихватив с собой генератор и
мобильник с \internet ом \smiley.

\emph{
Вполне можно ожидать, что эти тенденции приведут к появлению и оформлению нового
культурного течения, которое можно назвать} \textbf{скрэтчинг}\footnote{\ тут бы
хорошо подошло слово <<рукоблудие>>, но термин к сожалению уже занят, а
другого русского аналога подобрать пока не удалось}.

Этот журнал создан для \scr ов\ --- людей, чье хобби создавать вещи и технологии
по следам уже существующих, в сотый раз изобретать велосипед, чтобы разобраться
как оно работает, научиться делать самому, а возможно найти новый или забытый
способ что-то сделать, и конечно получить удовольствие от процесса поиска.

}{}

