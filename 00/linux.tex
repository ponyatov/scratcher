
\article{Хрюникс}{Собираем Cross Linux}{

В качествe примера применения возьмем относительно простое приложение:
многофункциональные настенные часы, с синхронизацией времени через \internet, с
будильником, медиапроигрывателем, блэкджеком и плюшками.

\subarticle{Linux для встраиваемых систем}

Linux для встраиваемых систем\footnote{\ будем называть его \emlinux}\ ---
популярный метод быстрого создания комплекса ПО для больших сложных приложений,
работающих на достаточно мощном железе, особенно предполагающих интенсивное
использование сетевых технологий.

За счет использования уже существующей и очень большой базы исходных текстов
ядра, библиотек и программ для \linux, бесплатно доступных в т.ч. и для
коммерческих приложений, можно на порядки сократить стоимость разработки
собственных программных компонентов, и при этом получить очень мощную команду
бесплатных стронних разработчиков, уже знакомых с созданием ПО для \linux.

Из недостатков можно отметить:
\begin{itemize}
  \item Отсутствие полноценной поддержки режима жесткого реального времени;
  \item Тяжелое ядро;
  \begin{itemize}
  \item Поддерживаются только мощные семейства процессоров;
  \item Значительные требования по объему \ram\ и общей производительности;
  \end{itemize}
  \item Дремучесть техспециалистов, контуженных ТурбоПаскалем и
Win\-dows\-ом;
\end{itemize}

Для \emph{сборки}\ \emlinux-системы используется метод кросс-компиляции, когда
используется \emph{кросс-тулчейн}, компилирующий весь комплект ПО для компьютера
с другой архитектурой\footnote{\ типичный пример\ --- сборка ПО на ПК c
процессором Intel i7 для Raspberry Pi или планшета на процессоре
AllWinner/Tegra/\ldots}.

\emlinux\ очень широко применяется на рынке мобильных устройств\footnote{\ в
т.ч. является основой Android}, и ус\-тройств интенсивно использующих сетевые
протоколы (роутеры, медиацентры).

\subarticle{Требования к системе сборки (\file{BUILD})}

Требования жесткие\ --- 4х-ядерный процессор, 4+\,Гб \ram, 64х-битный
дистрибутив \linux\ (рекомедую Debian), и никаких виртуалок.

Возможна установка системы на флешку, в этом случае требования к \ram\ еще более
ужесточаются\ --- потребуется каталоги с временными файлами смонтировать как
\file{tmpfs}.

Сборка под MinGW/Cygwin совершенно неживая. Если совсем никак без винды\ ---
используйте виртуалки, и будьте готовы ждать часами.

Можно попытаться сделать \file{билд-сервер}\ и на худшем железе, но будьте
готовы к тормозам или внезапному окончанию памяти\ --- ресурсоемка сборка
тяжелых библиотек типа \file{libQt} или крупных пакетов типа \file{gcc}.

В этом номере \Scr а описана сборка только базовой системы. Вы можете
попробовать поставить \linux\ на виртуалку, на флешку, и на жесткий диск (если
найдете место) и оценить возможности этих вариантов на сборке пакета \file{gcc}.

\subarticle{\file{Makefile}\ и пакеты}

Сборка выполняется утилитой \file{make}, описание структуры проекта для которой
прписано в файлах \file{Makefile} и \file{*.mk}.

В обычных дистрибутивах пакетами называют архивы скомпилированных программ,
устанавливаемых в \linux-систему. В нашем случае кросс-компиляции, будем
называть \term{пакетом}\ архив исходных текстов определенной программы или
компонента системы, вместе с секциеями мейкфайла, выполняющего ее компиляцию.

Также пакет может быть не связан с компиляцией ПО, а выполнять какую-то
вспомогательную работу.

Пакеты запускаются по своему имени вручную с помощью команды
\lstinputlisting{00/makesample.rc}

\subarticle{Файловая структура проекта}

Получите последнюю верию системы \file{cross}\ командой:

\lstinputlisting[title=git clone]{00/gitclone.rc}

\begin{tabular}{l l}
gz/ & локальное зеркало архивов исходных текстов \\
src/ & временный каталог рапаковки исходников \\
tmp/ & каталог out-of-tree сборки пакетов \\
build/ & каталог установки \$(BUILD)-части пакетов \\
\$(TARGET)/ & \file{rootfs}\ целевой системы \\
user/ & \\
cross/mk.rc & \\
\end{tabular}

\subsubsection{\file{mk.rc}: скрипт генерации \file{Makefile}}

Главный \file{Makefile}\ генерируется по частям скриптом:

\lstinputlisting[title=mk.rc]{../cross/mk.rc}

Это было сделано для удобства вставки частей (.mk файлы) в документацию, в т.ч.
и в эту статью.

\subsubsection{cross/head.mk}

\begin{itemize}
  \item{\file{HW}} железка, на которой будет запускаться
  \item{\file{APP}} приложение 
\end{itemize}

Во вложенных файлах прописываются переменные:

\begin{itemize}
  \item{\file{ARCH}} архитектура железки
  \item{\file{CPU}} процессор и 
  \item{\file{TARGET}} \term{триплет целевой платформы} 
\end{itemize}

\lstinputlisting[title=head.mk]{../cross/head.mk}

\lstinputlisting[title=config/hw/qemu386.mk]{../cross/config/hw/qemu386.mk}

\lstinputlisting[title=config/arch/i386.mk]{../cross/config/arch/i386.mk}

\lstinputlisting[title=config/cpu/i486.mk]{../cross/config/cpu/i486.mk}

\subsubsection{\file{dirs}: создание рабочих каталогов}

\begin{itemize}
  \item[\file{GZ}] зеркало архивов исходных текстов
  \item[\file{SRC}] сюда распаковываются исходные тексты
  \item[\file{TMP}] каталог для out-of-tree сборки 
  \item[\file{BUILD}] кросс-компилятор и другие программы \file{BUILD}-системы
  \item[\file{ROOT}] целевая файловая система \file{rootfs} (initrd)
  \item[\file{BOOT}] файлы, относящиеся к прцессу загрузки
\end{itemize}

Пакет \file{dirs}\ создает дерево каталогов. 

\lstinputlisting[title=dirs.mk]{../cross/dirs.mk}

\subsubsection{версии пакетов}

В этой части прописаны версии пакетов.

\lstinputlisting[title=versions.mk]{../cross/versions.mk}

\subsubsection{пакеты}

Здесь прописаны полные имена пакетов вместе с версиями. 

\lstinputlisting[title=packages.mk]{../cross/packages.mk}

\subsubsection{команды}

Макросы команд, используются далее:

\lstinputlisting[title=commands.mk]{../cross/commands.mk}

\subsubsection{\file{*clean}: очистка дерева проекта}

Для каталогов, прописанных в \file{/etc/fstab}\ как \file{tmpfs}\ (файловая
система в \ram) для экономии ресурса флешки и максимального ускорения сборки:

\lstinputlisting[title=/etc/fstab]{00/fstab.txt}

существует пакет \file{ramclean}, выполняющий их очистку.

Если вы используете такие каталоги, добавляете его вызов в конце вторым+ 
пакетом, чтобы освободить \ram\ для следующей сборки. 

\lstinputlisting[title=clean.mk]{../cross/clean.mk}

\subsubsection{\file{debian}: установка пакетов Debian Linux}

Перед сборкой нужно поставить несколько пакетов разработчика:

\lstinputlisting[title=debian.mk]{../cross/debian.mk}

\subsubsection{набор макросов для \file{configure}}

\begin{itemize}
  \item{\file{CFG}} общая часть запуска \file{configure}
  \item{\file{BCFG}} для \file{BUILD}-пакетов
  \item{\file{TCFG}} для \file{TARGET}-пакетов
\end{itemize}

\begin{itemize}
  \item{\file{--disable-nls}} диагностические сообщения и документация только на
английском
\item{\file{--prefix}} каталог установки скомпилированного пакета
\end{itemize}

\lstinputlisting[title=cfg.mk]{../cross/cfg.mk}

\subsubsection{\file{gz}: зеркалирование исходников}

Пакет \file{gz}\ выполняет загрузку из \internet а архивов исходных текстов
пакетов. 

Длительная и потребляющая трафик операция, нужен онлайн. По-хорошему архив
исходников тут было бы желательно загружать через пиринговые сети, а не
нагружать зеркала.

\lstinputlisting[title=gz.mk]{../cross/gz.mk}

\subsubsection{правила распаковки архивов исходников}

\lstinputlisting[title=srcrules.mk]{../cross/srcrules.mk}

\subarticle{Сборка \file{BUILD}-части}

\subsubsection{\file{binutils}: ассемблер и линкер}

Пакет \file{binutils}\ включает ассемблер, линкер и вспомогательные программы
для работы с объектными файлами в формате \term{ELF}. 

\begin{itemize}
  \item{\file{--target}} триплет целевой системы 
  \item{\file{--with-sysroot}} каталог с include/lib файлами целевой системы
  \item{\file{--with-native-system-header-dir}} относительно \file{SYSROOT}
\end{itemize}

\lstinputlisting[title=binutils.mk]{../cross/binutils.mk}

\subsubsection{\file{cclibs}: библиотеки поддержки GCC}

\lstinputlisting[title=cclibs.mk]{../cross/cclibs.mk}

\subsubsection{\file{gcc}: сборка компилятора GCC}

\lstinputlisting[title=gcc.mk]{../cross/gcc.mk}

\subarticle{Сборка базовой целевой системы}

\subsubsection{\file{kernel}: ядро Linux}

\lstinputlisting[title=kernel.mk]{../cross/kernel.mk}

\subsubsection{\file{libc}: главная библиотека uClibc}

\lstinputlisting[title=libc.mk]{../cross/libc.mk}

\subsubsection{\file{bb}: пакет UNIX-утилит BusyBox}

\lstinputlisting[title=busybox.mk]{../cross/busybox.mk}

\subsection{\file{root}: Генерация файловой системы}

\subsubsection{\file{boot}: сборка загрузчика}

\lstinputlisting[title=boot.mk]{../cross/boot.mk}

\subsubsection{\file{user}: сборка пользовательского ПО}

\lstinputlisting[title=user.mk]{../cross/user.mk}

\subsection{\file{qemu}: Запуск готовой системы в эмуляторе QEMU}

\subsubsection{cross/qemu.mk}

На этом этапе мы имеем только базовую систему, которую можно запустить в
эмуляторе, или на реальном x86-ом компьютере.

\lstinputlisting[title=qemu.mk]{../cross/qemu.mk}

\subsection{Конфигурирование ядра}

Отдельно рассмотрим опции конфигурации ядра. Эта информация поможет вам
адаптировать конфигурацию ядра для вашей конкретной железки, создав собственные
файлы конфигурации \file{config/hw/*.kernel}, \file{config/arch/*.kernel},
\file{config/cpu/*.kernel}.

Формирование результирующего конфигурационного файла см. \file{mk.rc}\ и секцию
мейкфайла для пакета \file{kernel}.

\subsection{Опции общие для всех вариантов сборки}

\subsubsection{\file{rootfs}\ в \ram}

\lstinputlisting[title=config/ramdisk.kernel]{../cross/config/ramdisk.kernel}

\subsubsection{режим реального времени}

\lstinputlisting[title=config/realtime.kernel]{../cross/config/realtime.kernel}

\subsubsection{часы и таймеры}

\lstinputlisting[title=config/clock.kernel]{../cross/config/clock.kernel}

\subsubsection{вывод \file{printk}\ и отладка}

\lstinputlisting[title=config/debug.kernel]{../cross/config/debug.kernel}

\subsubsection{форматы исполняемых файлов}

\lstinputlisting[title=config/binformats.kernel]{../cross/config/binformats.kernel}

\subsubsection{носители данных}

\lstinputlisting[title=config/storage.kernel]{../cross/config/storage.kernel}

\subsubsection{файловые системы}

\lstinputlisting[title=config/filesystems.kernel]{../cross/config/filesystems.kernel}

\subsubsection{интернационализация}

\lstinputlisting[title=config/i10n.kernel]{../cross/config/i10n.kernel}

\subsubsection{мультимедиа}

\lstinputlisting[title=config/media.kernel]{../cross/config/media.kernel}

\subsection{Опции для архитектуры \file{i386}}

\lstinputlisting[title=config/arch/i386.kernel]{../cross/config/arch/i386.kernel}

\subsubsection{опции процессора \file{i486}}

\lstinputlisting[title=config/cpu/i486.kernel]{../cross/config/cpu/i486.kernel}

\subsubsection{опции для эмулятора \file{qemu386}}

\lstinputlisting[title=config/hw/qemu386.kernel]{../cross/config/hw/qemu386.kernel}


}{}
