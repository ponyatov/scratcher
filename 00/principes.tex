
\article{Принципы скрэтчера}{Чем оно отличается от прочего DIY}{

\begin{itemize}

\item Из говна и палок

Чем больше г и кривее палки, тем круче \scr

\item Сделай сам, расскажи другим

Необходим активный обмен информацией для мимимизации и так больших расходов на
избретение колес

\item Минимум покупных изделий

В идеале изготовление всего из чисто природных материлов и без стартового
инструмента

\item Все покупные ништяки должны быть всегда доступны в любом ближайшем
магазине

Чтобы каждый мог легко и быстро повторить понравившийся хак.

\emph{Следует обратить внимание, что этому принципу противоречит использование
техно-мусора, различных деталей от старой техники и т.п.\ --- вот сколько
сейчас у вас например сломанных стиралок, или дохлых телевизоров в доме ?}

\emph{Еще одно противоречие\ --- покупка комплектухи по почте в Китае, и заказ
редких компонентов в магазинах}

\item Покупаться должны \textbf{самые дешевые}\ и самые кривые
комплектующие

Но при этом не нужно скатываться на использование раритета\ --- см. доступность.

\item Приоритетно использование более ранней ступени
\term{технологического передела}

Например вместо использование готового заводского сверла взять хвостовик от
сломанного, и выпилить сверло самому. Правильнее было бы взять твердосплавную
заготовку, но это противоречит принципу доступности, т.к. их нет в
доступных магазинах. Вариант использование куска проката из инструментального
сплава лучше, потому что можно еще повыделываться с термичкой \smiley

\item Должно использоваться \term{открытое программное обеспечение}

Причем написанное целиком самостоятельно на ассемблере, ну или хотя бы собрать
\href{http://cross-lfs.org/}{Cross Linux From Scratch}\ для DIY компьютера,
спаянного из отдельных деталей с помощью самодельного паяльника.

В процессе неплохо попутно изобрести пару уникальных языков программирования,
написать на них операционную систему и комплект программного обеспечения.

\item Желательно использовать нетиповые приемы работы и технологии

\item При разработке конструкций нужно стремиться использовать малоизвестные и
уникальные конструктивные решения

\item Максимум самодельного инструмента

\item Идеал \scr а\ --- пройти всю технологическую цепочку от каменного
рубила до обрабатывающего центра с ЧПУ

И с разгона заскочить еще дальше, обогнав текущие лабораторные разработки по
3D-печати, зональной плавке и прочим свежакам технологии

\item Минимум повторов готовых изделий и унификации

Каждая поделка должна быть прекрасна в своей уникальности, и ее область
применения должна быть максимально узкозаточенной под ваши задачи. Применение
унификации, общеизвестных конструктивных решений и принципов работы неприемлемо,
т.к. какой смысл повторять уже готовое изделие, которое можно купить ?!

\item Больше науки

Копайте книги по математике, физике и химии, больше статей и техрасчетов.
Чем больше матана и самопала, тем выше левел. Не забывайте про пропагандизм
достижений на форумах (особенно нетематических) и в оффлайне.

\item Больше синей изоленты

\item Обязательно используйте ардуину

Даже если устройство вообще не предполагает использование электричества\ ---
прикрутите микроконтроллер изолентой, и подключите к нему компьютер.

\item Для успеха проекта обязательно нужен ковер

\item На демонстрационном видео должно что-нибудь отвалиться или чпохнуть
волшебным синим дымом

\end{itemize}

Набор принципов \scr а выглядит похоже на инструкцию <<Как просрать полимеры>>,
поэтому как и в любом другом деле, не нужно доводить их исполнение до фанатизма.
\emph{Новизна и уникальность} на первом месте, и не надо забывать что это все же
хобби, а не жизненная миссия. Нужно всего лишь следить за соблюдением баланса
между потраченными средствами, временем, и полученным от процесса удовольствием.

Главное достоинство отработанных вещей и технологий\ --- на их доводку и
проверку уже было потрачено гигантское количество ресурсов. Самодельные аналоги
в любом случае будут хуже и на порядок дороже, чем серийное изделие, за редким
исключением узконишевого использования, для которого готовое решение почему-то
не подходит.

Из положительных эффектов \scr ства можно отметить хорошие общетехнические
знания, и умение при необходимости быстро слепить <<костыль>> (временное решение
проблемы) из подручных ресурсов.

}{}

